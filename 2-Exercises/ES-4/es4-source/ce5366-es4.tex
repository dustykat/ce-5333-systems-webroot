\documentclass[12pt]{article}
% REVISION NOTES %%%%%%%%%%%%%%%%%%%%%%%%%%%%%%%%%%%%%%%%%%%%
% 2008-0814 Location, Date, Time
% 2008-0814 fixed citations -- added bibliography.
%
%
%\usepackage{geometry}                
%\geometry{letterpaper}                   
%\geometry{landscape}                
\usepackage[parfill]{parskip}    
\usepackage{daves,fancyhdr,natbib,graphicx,dcolumn,amsmath,lastpage,url}
\usepackage{amsmath,amssymb,epstopdf,longtable}
\usepackage[final]{pdfpages}
\usepackage{paralist} 
\DeclareGraphicsRule{.tif}{png}{.png}{`convert #1 `dirname #1`/`basename #1 .tif`.png}
\pagestyle{fancy}
\lhead{CE 5366}
\rhead{FALL 2018}
\lfoot{CE 5333 -- Cleveland}
\cfoot{Page \thepage\ of \pageref{LastPage}}
\rfoot{REVISION NO. 1}
\renewcommand\headrulewidth{0pt}

%%%%%%%%%% Will's stuff BELOW %%%%%%%
\usepackage[left=1.25in, right=1.25in,
            top=1in, bottom=1in]{geometry}                % See geometry.pdf to learn the layout options. There are lots.
\geometry{letterpaper}

\usepackage{ragged2e}

\usepackage{xcolor}
\newcommand{\codeRcolor}{0.93}
\newcommand{\codeGcolor}{0.93}
\newcommand{\codeBcolor}{0.93}
\definecolor{lightgrey}{rgb}{\codeRcolor,
                             \codeGcolor,
                             \codeBcolor}

\newcommand{\listingfont}{\fontsize{7pt}{8pt}\selectfont\ttfamily}
\usepackage{listings}
\lstset{basicstyle = \listingfont,
        breaklines = true,
        frame=tb,
        xleftmargin=12pt,
        framexleftmargin=6pt,
        framexrightmargin=6pt,
        xrightmargin=12pt,
        columns=fixed}
\lstset{lineskip=-1pt}
\lstset{backgroundcolor=\color{lightgrey}}


\usepackage[font={footnotesize},
            labelfont={sf,bf},
            textfont={sf},
            singlelinecheck=false,
            labelsep=none,
            justification=RaggedRight,
            aboveskip=0pt,
            belowskip=7pt plus 1pt minus 1pt,
            textformat=period]{caption}
\DeclareCaptionLabelSeparator{mystyle}{.\quad}
\captionsetup{labelsep=mystyle}
%%%%%%%%% Will's Stuff ABOVE %%%%%%%%%

%%%%%%%%%%%%%%%%%%%%%%%%%%%%%%%%%%%%%%%%%%%%%%%%%%%%%%%
\begin{document}
\begin{center}
{\textbf{{ CE 5366 -- Water Resources Management}  {Exercise Set 4}}}
\end{center}
\begingroup
\begin{tabular}{p{1in} p{5in}}
Purpose: & Engineering economic mathematics \\
\end{tabular}
\endgroup
%%%%%%%%%%%%%%%%%%%%%%%
\section*{\small{Exercise}}
\begin{enumerate}
\item Two mutually exclusive project alternatives that provide identical service are described below: \\~\\
\begin{tabular}{lrrrr}
Project ID & Initial Cost & Annual O\&M & Salvage Value & Lifespan \\
\hline
\hline
A & \$10,000 & \$2,000 & \$1,000 & 10  \\
B & \$25,000 & \$1,500 & \$5,000 & 20  \\
\end{tabular}\\~\\
Assuming a discount rate of 5\% and using the \textbf{R} script in Listing \ref{lst:AnnualCosts} (or write your own)\footnote{The script here is pretty crude.  User supplies a guess of annual costs, and by repeated application changes the guess until the computed present value of the annual costs is equal to the pre-determined present value based on the supplied components.  A vast improvement would be to make the guess-and-check automatic; Newton's method (finite-difference approximations to the derivative) or bisection would work well.}, determine:
\begin{enumerate}[a)]
\item Which alternative has the lower annual cost?
\item What is the incremental annual cost of going from the less to the more expensive alternative?
\item Select the best alternative by the present-worth method.
\item What is the rate of return on the incremental investment of B?
\item What initial cost of replacing A after 10 years would make the two alternatives equivalent, assuming none of the other costs change?

\end{enumerate}
\end{enumerate}

\begin{lstlisting}[caption=R code for Trial-Error to find equivalent annual cost, label=lst:AnnualCosts]
# cash flow calculations for ES4
rm(list=ls())
discount_rate <- 0.03 # 3% interest rate
# option A
# compute present values of expenses first 10 years
initial_costA <- 10000
operationsA <- rep(2000,10)
salvageA <- 1000
### compute the PV of the operations cost
present_value <- numeric(0)
present_value <- 0
for (i in 1:10){ #find present value of the i-th year payment
  present_value[i] <- operationsA[i]*(1+discount_rate)^(-i)
}
operationsAPV <- sum(present_value)
### compute the PV of the salvage payment
salvageAPV <- salvageA*(1+discount_rate)^(-10)
#print(cbind(initial_costA,operationsAPV,salvageAPV))
# now compute the PV for the second 10 years
presentValueA1 <- initial_costA+operationsAPV+salvageAPV
presentValueA2 <- presentValueA1*(1+discount_rate)^(-10)
presentValueA <- presentValueA1+presentValueA2
message("Present Value A = $",presentValueA)
########## Alternative B #################
initial_costB <- 25000
operationsB <- rep(1500,20)
salvageB <- 5000
### compute the PV of the operations cost
present_value <- numeric(0)
present_value <- 0
for (i in 1:20){ #find present value of the i-th year payment
  present_value[i] <- operationsB[i]*(1+discount_rate)^(-i)
}
operationsBPV <- sum(present_value)
### compute the PV of the salvage payment
salvageBPV <- salvageB*(1+discount_rate)^(-20)
#print(cbind(initial_costB,operationsBPV,salvageBPV))
presentValueB <- initial_costB+operationsBPV+salvageBPV
message("Present Value B = $",presentValueB)
# now convert both to annual costs
### compute the PV of the operations cost
present_valueA <- numeric(0)
present_valueA <- 0
### compute the PV of the operations cost
present_valueB <- numeric(0)
present_valueB <- 0
## read in guess for annual cost
  avA <- readline(prompt="Enter annual cost for alternative A: ")
  avA <- as.numeric(avA)
  avB <- readline(prompt="Enter annual cost for alternative B: ")
  avB <- as.numeric(avB)
annualA <- rep(avA,20)
annualB <- rep(avB,20)
for (i in 1:20){ #find present value of the i-th year payment
  present_valueA[i] <- annualA[i]*(1+discount_rate)^(-i)
  present_valueB[i] <- annualB[i]*(1+discount_rate)^(-i)
}
message("annual cost A = $",avA," PVA = $",sum(present_valueA)," PValue A = $",presentValueA)
message("annual cost B = $",avB," PVB = $",sum(present_valueB)," PValue B = $",presentValueB)



### CONSOLE INTERACTION ###
> source('~/Dropbox/1-Teaching/ce-5366/2-Exercises/ES-3o/cash-flow-ES3.R')
Present Value A = $48493.6585389613
Present Value B = $50084.5910616149
Enter annual cost for alternative A: 3259.5
Enter annual cost for alternative B: 3366.5
annual cost A = $3259.5 PVA = $48493.1293076547 PValue A = $48493.6585389613
annual cost B = $3366.5 PVB = $50085.0191177235 PValue B = $50084.5910616149
> \end{lstlisting}  



\end{document}


